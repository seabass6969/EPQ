\begin{longtable}{|p{4.5cm}|c|p{5cm}|p{2.5cm}|}
    \hline 
    Sources & & Critics & Notes \\
    \hline
    \fullcite{amandaghassaei_what_nodate} & \cite{amandaghassaei_what_nodate} & 
    This tutorial article is indented to provide a really basic definition about the MIDI file. 
    & \\
    \hline
    \fullcite{noauthor_fast_2025} & \cite{noauthor_fast_2025} &
    This Wikipedia is indented to provide a really basic definition about the fast Fourier Transform. As Wikipedia might not be the most credible source. And it might have inaccuracy. So this source is not used extensively.
    & \\
    \hline 
    \fullcite{ffmpeg_developers_ffmpeg_nodate} & \cite{ffmpeg_developers_ffmpeg_nodate} &
    This program is indented to be used in the program. This program extensively used in audio processing inside the code as an assistant / helper tool. This tool is also extensively used for other audio related processing. 
    & Utilized only in the code\\
    \hline 
    \fullcite{ghias_query_1995} & \cite{ghias_query_1995} &
    This conference article provide a discussion point on how to make the artefact if I want to do it differently. This source is credible to an extent as it's written by ACM which is a scientific and educational computing society. However, it might be slightly outdated (published date 1995) as there might be more information about this.
    & \\
    \hline 
    \fullcite{harris_array_2020} & \cite{harris_array_2020} &
    This program is indented to be used in the program. This program extensively used in data analysis. It's an industry standard tool to analysis array in all fields, such as research.
    & Utilized only in the code\\
    \hline 
    \fullcite{noauthor_hash_2024} & \cite{noauthor_hash_2024} &
    This Wikipedia is indented to use to provide a really basic definition about the hash function. As Wikipedia might not be the most credible source. And it might have inaccuracy. So this source is not used extensively.
    & \\
    \hline 
    \fullcite{macleod_abracadabra_nodate} & \cite{macleod_abracadabra_nodate} &
    This blog post is important because this includes much more details that the original article from the Shazam company. However, this does not mean this blog post are accurate as this is from written from a perspective of the original creator. But the writer is partially qualified to write about this topic as the writer work as a software engineer at Bloomberg and a graduate of University of Edinburgh.
    & \\
    \hline 
    \fullcite{newnham_interview_2023} & \cite{newnham_interview_2023} &
    This interview give an insight into the creator of Shazam Avery Wang's identity, which his source is used as concrete source in the program. \cite{wang_systems_2013, wang_industrial-strength_2003}
    & \\
    \hline 
    \fullcite{virtanen_scipy_2020} & \cite{virtanen_scipy_2020} &
    This program is indented to be used in the program. This program extensively used in solving mathematical problem and scientific tool. And it's an industry standard programming interface for solving mathematical problem and analysing scientific problem.
    & Utilized only in the code\\
    \hline 
    \fullcite{wang_industrial-strength_2003} & \cite{wang_industrial-strength_2003} &
This is the original article explaining from a prospective of the creator. This article creates a basic understanding of how the basics works. They have a detail analysis of recognition rate and the performance associated. However, the article is missing from explaining the method of how they create Spectrogram, and the algorithm they are using to create it. There might be changes to the modern day algorithm that the product is using, this might be due to the age of this paper being published in 2003. And since now the company are acquired by Apple Inc.
    & \\
    \hline 
    \fullcite{wang_systems_2013} & \cite{wang_systems_2013} &
    This source is very credible and concrete, as the original program Shazam are based on it. It is created by the founder and principle scientist of Shazam. Inside the source there are also figures attached to it. This patent is cited (as of February 2025) 771 times. It's being used in various different field, most notability social media companies and big tech companies such as Google, Snap. However, the source might not be as relevant because the patent are outdated and expired.
    & \\
    \hline 
    \fullcite{yang_music_2001} & \cite{yang_music_2001} &
    This source provides resources mainly how to test the artefact. This report are also written by the Stanford University which is a very credible source and provides great information.
    & \\
    \hline 
\end{longtable}
