\documentclass[twoside]{report}
\title{Template}
\author{Cephas Yeung}
\date{28/10/2024}

% MIT License

% Copyright (c) [2024] [Cephas Yeung]

% Permission is hereby granted, free of charge, to any person obtaining a copy
% of this software and associated documentation files (the "Software"), to deal
% in the Software without restriction, including without limitation the rights
% to use, copy, modify, merge, publish, distribute, sublicense, and/or sell
% copies of the Software, and to permit persons to whom the Software is
% furnished to do so, subject to the following conditions:

% The above copyright notice and this permission notice shall be included in all
% copies or substantial portions of the Software.

% THE SOFTWARE IS PROVIDED "AS IS", WITHOUT WARRANTY OF ANY KIND, EXPRESS OR
% IMPLIED, INCLUDING BUT NOT LIMITED TO THE WARRANTIES OF MERCHANTABILITY,
% FITNESS FOR A PARTICULAR PURPOSE AND NONINFRINGEMENT. IN NO EVENT SHALL THE
% AUTHORS OR COPYRIGHT HOLDERS BE LIABLE FOR ANY CLAIM, DAMAGES OR OTHER
% LIABILITY, WHETHER IN AN ACTION OF CONTRACT, TORT OR OTHERWISE, ARISING FROM,
% OUT OF OR IN CONNECTION WITH THE SOFTWARE OR THE USE OR OTHER DEALINGS IN THE
% SOFTWARE.
% 
% GitHub link:
% https://github.com/seabass6969/latex-template
\usepackage[utf8]{inputenc}

\usepackage{amsmath, amsfonts, amsthm, graphicx, geometry, lipsum, subfigure}
\usepackage{blindtext}
\makeatletter
\let\runauthor\@author
\let\runtitle\@title
\makeatother

\usepackage{pdfpages}
\usepackage{appendix}

\usepackage[nottoc, numbib]{tocbibind}
% \usepackage[comma]{natbib}
\usepackage[
  backend=biber,
  style=numeric, % default
  % style=alphabetic,
  % style=authoryear,
]{biblatex}

\usepackage{verbatim, fancyvrb}
\usepackage{fancyhdr, lastpage}
\pagestyle{fancy}
\lhead{\runtitle}
\rhead{\runauthor}
\usepackage{listings}


% \patchcmd{\chapter}{\thispagestyle{plain}}{\thispagestyle{fancy}}{}{}
% \expandafter\patchcmd\csname chapter*\endcsname{\thispagestyle{plain}}{\thispagestyle{fancy}}{}{}
\usepackage[Sonny]{fncychap}
\usepackage{xcolor-material}
\usepackage[svgnames]{xcolor}
\usepackage{tikz, tcolorbox}
\tcbuselibrary{listings,breakable}
\usepackage{longtable}
\usepackage{hyperref}

\hypersetup {
    colorlinks=true,
    linkcolor=blue,
    urlcolor=red,
    citecolor=RoyalBlue
}
\renewcommand\theHtable{Appendix.\thetable}

\newcommand{\trademark}{\textsuperscript{\textregistered}}
\newcommand\TBox[3][]{%
  \tikz\node[draw,ultra thick,text width=#2,align=left,#1] {#3};}


% \usepackage{harvard}
\patchcmd{\thebibliography}{\chapter*}{\section}{}{}

\usetikzlibrary{matrix}
\definecolor{15}{HTML}{000000}
\definecolor{14}{HTML}{111111}
\definecolor{13}{HTML}{222222}
\definecolor{12}{HTML}{333333}
\definecolor{11}{HTML}{444444}
\definecolor{10}{HTML}{555555}
\definecolor{9}{HTML}{666666}
\definecolor{8}{HTML}{777777}
\definecolor{7}{HTML}{888888}
\definecolor{6}{HTML}{999999}
\definecolor{5}{HTML}{AAAAAA}
\definecolor{4}{HTML}{BBBBBB}
\definecolor{3}{HTML}{CCCCCC}
\definecolor{2}{HTML}{DDDDDD}
\definecolor{1}{HTML}{EEEEEE}
\definecolor{0}{HTML}{FFFFFF}

\tikzset{ 
table/.style={
  matrix of nodes,
  row sep=-\pgflinewidth,
  column sep=-\pgflinewidth,
  nodes={rectangle,draw=black,fill=0,minimum size=1cm,align=center},
  nodes in empty cells
  }
}
\tikzset{
    block/.style ={ rectangle, 
draw, fill=white,  text centered, minimum size=1cm, nodes in empty cells,
  row sep=-\pgflinewidth,
  column sep=-\pgflinewidth,
fill=green}
    }

\usetikzlibrary{shapes.misc}
\usepackage{pgfplots}
\pgfplotsset{width=7cm,compat=1.8}


\usepackage{longtable}
\begin{document}
\begin{itemize}
\item
  \phantomsection\label{item_S93N584M}
  \subsection{Euler\textquotesingle s formula}\label{eulers-formula}

  \begin{longtable}[]{@{}ll@{}}
  \toprule\noalign{}
  \endhead
  \bottomrule\noalign{}
  \endlastfoot
  Item Type & Encyclopedia Article \\
  Date & 2024-09-26T07:05:39Z \\
  Language & en \\
  Library Catalog & Wikipedia \\
  URL &
  \url{https://en.wikipedia.org/w/index.php?title=Euler\%27s_formula&oldid=1247828798} \\
  Accessed & 11/7/2024, 10:15:21 PM \\
  Rights & Creative Commons Attribution-ShareAlike License \\
  Extra & Page Version ID: 1247828798 \\
  Encyclopedia Title & Wikipedia \\
  Date Added & 11/7/2024, 10:15:21 PM \\
  Modified & 11/7/2024, 10:43:19 PM \\
  \end{longtable}

  \subsubsection{Notes:}\label{notes}

  \begin{itemize}
  \item
    \phantomsection\label{item_ZMHWSIRK}
    This is useful for learning about the basics behind the Fourier
    transform function as a backing knowledge
  \end{itemize}

  \subsubsection{Attachments}\label{attachments}

  \begin{itemize}
  \tightlist
  \item
    \phantomsection\label{item_UTDLGEQL}{Snapshot}
  \end{itemize}
\item
  \phantomsection\label{item_INLNF82T}
  \subsection{An Industrial-Strength Audio Search
  Algorithm}\label{an-industrial-strength-audio-search-algorithm}

  \begin{longtable}[]{@{}ll@{}}
  \toprule\noalign{}
  \endhead
  \bottomrule\noalign{}
  \endlastfoot
  Item Type & Journal Article \\
  Author & Avery Li-Chun Wang \\
  Date & January 2003 \\
  Language & en \\
  Library Catalog & Zotero \\
  Date Added & 10/28/2024, 8:33:16 PM \\
  Modified & 11/7/2024, 10:26:10 PM \\
  \end{longtable}

  \subsubsection{Notes:}\label{notes-1}

  \begin{itemize}
  \item
    \phantomsection\label{item_4C2M86CP}
    \section{This is the original article explaining from a prospective
    of the
    creator}\label{this-is-the-original-article-explaining-from-a-prospective-of-the-creator}

    This article create a basic understanding of how the basics works.

    They have a detail analysis of recognition rate and the performance
    associated.

    However the article are missing from explaining the method of how
    they create Spectrogram, and the algorithm they are using to create
    it.

    There might be changes to the modern day algorithm that the product
    is using, this might be due to the age of this paper being published
    in 2003. And since now the company are acquired by Apple Inc.
  \end{itemize}

  \subsubsection{Attachments}\label{attachments-1}

  \begin{itemize}
  \tightlist
  \item
    \phantomsection\label{item_RQ6E6ET6}{PDF}
  \end{itemize}
\item
  \phantomsection\label{item_J7GZJFWB}
  \subsection{abracadabra: How does Shazam work? - Cameron
  MacLeod}\label{abracadabra-how-does-shazam-work---cameron-macleod}

  \begin{longtable}[]{@{}ll@{}}
  \toprule\noalign{}
  \endhead
  \bottomrule\noalign{}
  \endlastfoot
  Item Type & Web Page \\
  Author & Cameron MacLeod \\
  Language & en \\
  Short Title & abracadabra \\
  URL &
  \url{https://www.cameronmacleod.com/blog/how-does-shazam-work} \\
  Accessed & 10/28/2024, 8:31:34 PM \\
  Date Added & 10/28/2024, 8:31:34 PM \\
  Modified & 10/28/2024, 8:31:34 PM \\
  \end{longtable}

  \subsubsection{Notes:}\label{notes-2}

  \begin{itemize}
  \item
    \phantomsection\label{item_4QPJSP4E}
    \section{Extension article to Original Article as a helpful
    tips}\label{extension-article-to-original-article-as-a-helpful-tips}
  \end{itemize}

  \subsubsection{Attachments}\label{attachments-2}

  \begin{itemize}
  \tightlist
  \item
    \phantomsection\label{item_LYURUIEI}{Snapshot}
  \end{itemize}
\end{itemize}
\end{document}
